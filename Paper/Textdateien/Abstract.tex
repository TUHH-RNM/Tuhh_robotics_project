This paper presents a method, where a robot is used to precisely compensate the body movements of a human patient during medical treatments. The method uses visual tracking to determine the position and orientation of the relevant body parts of the patient and afterward, based on the received tracking data, the robot is  moving a specific medical device, such that the device attains a commanded spatial relation to the relevant body parts of the patient. In the 
first section of this paper, it is shown, why such a method is useful and which fields of application exists. Furthermore, a specific medical scenario is presented, which will be used in this paper to demonstrate the motion compensation method. The second section presents the 
used robot and his kinematic model. The third section introduces the visual tracking system and the calibration and tracking algorithms to estimate the posture of the patient's body. The 
fourth section considers the constraints and conditions the motion of the robot is subject to and derives appropriate path planning algorithms to fulfill them. The fifth section shows how the actual motion compensation is assembled from the previous contents and the last section summarizes the results.