\section{Introduction}
\IEEEPARstart{I}{n} a number of medical applications, it is necessary to place a medical device 
next to a human patient in such way that the device keeps exactly a predetermined position and 
orientation relative to the body of the patient. One example is the placement of a radiation 
coil in space, next to the head of a patient, such that it can radiates a specific part of 
the patient's head. Naturally, a human patient is doing very small movements of his body parts, unconsciously, and 
therefore the considered device must follow this movements to maintain the commanded relation. 
Obviously, a robotic operator is suitable for executing this task. A tracking system measures 
the movements and posture of the patient and it does the same for the medical device. The 
tracking system sends this data to a PC, which processes the data, to determine which 
movements the end-effector of the robot must execute to maintain the specified spatial relation 
between the medical device and the patient. Then the PC sends movement commands to the robot 
and the robot is reacting with a movement of his manipulator, which carries the medical 
device. Of course, care must be taken that the robot or the device is not colliding with the 
patient. To demonstrate the above explanations the scenario shown in \figref{fig:scenario} is considered. 
\begin{figure}
\centering
\input{Bilder/scenario.pdf_tex}
\caption{Medical scenario, where the motion compensation method is used.}
\label{fig:scenario}
\end{figure}
The robot and the tracking system are shown which have their own coordinate systems attached to them. In this scenario, the only important body part of the patient is his head. The coil, which is carried by the end-effector of the robot, shall keep a predetermined posture relative to the head. To express this posture, the head and the coil also have corresponding coordinate systems attached to them. Then the relative posture between them can be expressed with the homogeneous transformation matrix $^{\text{C}}T_{\text{H}}$. Analogously, the \textit{desired} posture between head and coil has the corresponding homogeneous transformation matrix $^{\text{C}}T_{\text{H,g}}$. The goal of the motion compensation method is to move the robot, such that the relation
\begin{align}
^{\text{C}}T_{\text{H}} = ^{\text{C}}T_{\text{H,g}}
\label{eq:desired_relation}
\end{align}
always holds. Of course, due to measurement errors of the tracking system and the finite precision of the robot, $^{\text{H}}T_{\text{C}}$ can only approximate $^{\text{C}}T_{\text{H,g}}$. The coordinate frames of the patient's head, tracking system, robot and coil
have in each case a posture relative to the world coordinate frame, denoted by the axes triple
$\left( x_{\text{W}},y_{\text{W}},z_{\text{W}} \right)$. The tracking system measures the relative postures of the head relative to the coordinate system of the tracking system $\left( x_{\text{S}},y_{\text{S}},z_{\text{S}} \right)$, in terms of the homogeneous transformation matrices $^{\text{S}}T_{\text{H}}$. Note, that there are two kinds of tracking systems. The first tracking system is a Microsoft KINECT camera. It only outputs images with depth information to the PC. Based on the data of the KINECT, it is necessary to estimate $^{\text{S}}T_{\text{H}}$. This is done in the motion compensation method itself. The second available tracking system is an Atracsys FusionTrack tracking camera. Its output is compiled to $^{\text{S}}T_{\text{H}}$ by an independent subroutine running on the PC, which means that the motion compensation algorithm has directly access to $^{\text{S}}T_{\text{H}}$. This homogeneous transformation matrix  is processed together with the posture of the tracking system and of the robot in the world coordinate system to output data for the robot. This data processing is done in \textsc{Matlab}. The robot will then, based on the output data from the PC, move its end-effector, such that \eqref{eq:desired_relation} is approximated as good as possible.